\documentclass{article}
\usepackage{booktabs}
\usepackage{fancyhdr}
\pagestyle{empty}
\usepackage{graphicx}
 \usepackage{color}
 \usepackage{transparent}
 \usepackage{hyperref}
 \usepackage{lastpage}

\pagestyle{fancy}
\lhead{} % Top left header
\chead{Heqet Tutorial} % Top center head
\rhead{} % Top right header
\lfoot{} % Bottom left footer
\cfoot{} % Bottom center footer
\rfoot{ \thepage/\protect\pageref{LastPage}} % Bottom right footer \protect\pageref{LastPage}

\newcommand{\prob}[1]{\noindent\textbf{#1.}}

\begin{document}
\pagestyle{empty}
\begin{center}
\vspace{3cm}
\scalebox{1.5}{{\Huge Heqet Tutorial}}

\vspace{2cm}
{\Large By Isaac Reilly}

\vspace{0.15cm}
{\large advised by Donya Quick}

\vspace{0.1cm}
{senior project for the Yale University major in Computer Science}

\vspace{13cm}

{\small fall 2015}

\end{center}

\pagebreak

\tableofcontents
\newpage

\begin{section}{Heqet for Euterpea Users}

\end{section}

\begin{section}{Heqet for Lilypond Users}

\begin{subsection}{Note input}

\end{subsection}

\begin{subsection}{Differences from Lilypond input}
There are several differences between the Heqet note-input domain-specific language and Lilypond input, for a variety of reasons. 
\begin{itemize}
\item You can enter notes with any rational duration with the \verb+\d+ syntax, for example \verb+c\d 4/5+ to make a note with a duration of $4/5$ of a WHAT. You can omit the denominator if it's $1$. This is currently the only way to enter notes of the durations needed for a tuplet. 

\item \verb+hz+

\item \verb+c-(+ for slurs (sorry)

\item no clefs or time signatures

\item functions, commands, \verb+\with+ . Lilypond parsing problems

\item only absolute entry at the moment. 

\item percussion notation \verb+\phh+ for hi-hat. When you enter notes, you don't need to specify that they must be rendered in a DrumStaff. They should be automatically rendered well, although support for percussion is currently minimal.

\item no way to manually write beams, as this is not something the musician should need to worry about.
\end{itemize}
\end{subsection}

\begin{subsection}{Making a score}

\end{subsection}

\begin{subsection}{Lilypond tweaks}

\end{subsection}

\end{section}

\begin{section}{Advanced topics}

\end{section}

\end{document}